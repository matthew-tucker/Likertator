\documentclass[11pt, a4paper]{article}

\usepackage{MnSymbol}
\usepackage{mathrsfs}
\usepackage{mathspec,xunicode,xltxtra}

\usepackage{fontspec}

\defaultfontfeatures{Ligatures=TeX}
\setallmainfonts[Renderer=ICU]{Charis SIL}


\usepackage[top=1in,bottom=1in,left=1in,right=1in]{geometry}
\usepackage{fancyhdr}

\renewcommand\headrulewidth{0pt}
\pagestyle{fancy}
\fancyhead{}
\fancyhead[L]{{\footnotesize\textsc{List \# {1}}}}
\fancyhead[R]{{\footnotesize\textsc{Al-Bahrain}}}
\fancyfoot{}
\fancyfoot[C]{{\footnotesize Page \textbf{\thepage}\ of \textbf{\pageref{LastPage}}}}
\renewcommand{\headrulewidth}{0.4pt}
\addtolength{\headheight}{2.5pt}

\usepackage{lastpage}
\usepackage{etex}

\usepackage{fontspec}
\usepackage{polyglossia}

\setmainlanguage{english}
\setotherlanguage{arabic}

\newfontfamily\arabicfont[Script=Arabic,Scale=1.1]{Baghdad}

\usepackage{gb4e}
\noautomath

%%%%%%%%%%%%%%%%%%%%%%%%% PREAMBLE END %%%%%%%%%%%%%%%%%%%%%%%%%%%%%%%%%%%%

%%%%%%%%%%%%%%%%%%%%%%%%% BODY %%%%%%%%%%%%%%%%%%%%%%%%%%%%%%%%%%%%

\begin{document}
	
	{\begin{flushright}
\textarabic{من تساءل أي صورة هذه للبيع؟}
\end{flushright}

\begin{center}
        \hfill\textarabic{٧}\hfill\textarabic{٦}\hfill\textarabic{٥}\hfill\textarabic{٤}\hfill\textarabic{٣}\hfill\textarabic{٢}\hfill\textarabic{١}
        \end{center}


\vspace{0.5\baselineskip}\begin{flushright}
\textarabic{كان هناك رجل معتبر مريضاً}
\end{flushright}

\begin{center}
        \hfill\textarabic{٧}\hfill\textarabic{٦}\hfill\textarabic{٥}\hfill\textarabic{٤}\hfill\textarabic{٣}\hfill\textarabic{٢}\hfill\textarabic{١}
        \end{center}


\vspace{0.5\baselineskip}\begin{flushright}
\textarabic{دُخل النادي من دون حذاء}
\end{flushright}

\begin{center}
        \hfill\textarabic{٧}\hfill\textarabic{٦}\hfill\textarabic{٥}\hfill\textarabic{٤}\hfill\textarabic{٣}\hfill\textarabic{٢}\hfill\textarabic{١}
        \end{center}


\vspace{0.5\baselineskip}\begin{flushright}
\textarabic{من يدّعى أن الأصدقاء تبادلوا التحية في السوق؟}
\end{flushright}

\begin{center}
        \hfill\textarabic{٧}\hfill\textarabic{٦}\hfill\textarabic{٥}\hfill\textarabic{٤}\hfill\textarabic{٣}\hfill\textarabic{٢}\hfill\textarabic{١}
        \end{center}


\vspace{0.5\baselineskip}\begin{flushright}
\textarabic{من سمع أن إلياس أخذ الجائزة؟}
\end{flushright}

\begin{center}
        \hfill\textarabic{٧}\hfill\textarabic{٦}\hfill\textarabic{٥}\hfill\textarabic{٤}\hfill\textarabic{٣}\hfill\textarabic{٢}\hfill\textarabic{١}
        \end{center}


\vspace{0.5\baselineskip}\begin{flushright}
\textarabic{أعطى كتاب محمد}
\end{flushright}

\begin{center}
        \hfill\textarabic{٧}\hfill\textarabic{٦}\hfill\textarabic{٥}\hfill\textarabic{٤}\hfill\textarabic{٣}\hfill\textarabic{٢}\hfill\textarabic{١}
        \end{center}


\vspace{0.5\baselineskip}\begin{flushright}
\textarabic{يبدو على جميع الصبية أن انهوا جميع واجباتهم}
\end{flushright}

\begin{center}
        \hfill\textarabic{٧}\hfill\textarabic{٦}\hfill\textarabic{٥}\hfill\textarabic{٤}\hfill\textarabic{٣}\hfill\textarabic{٢}\hfill\textarabic{١}
        \end{center}


\vspace{0.5\baselineskip}\begin{flushright}
\textarabic{يعزف رامي الجيتار، و تظن سلمى أن باسم يفعل ذلك أيضا}
\end{flushright}

\begin{center}
        \hfill\textarabic{٧}\hfill\textarabic{٦}\hfill\textarabic{٥}\hfill\textarabic{٤}\hfill\textarabic{٣}\hfill\textarabic{٢}\hfill\textarabic{١}
        \end{center}


\vspace{0.5\baselineskip}\begin{flushright}
\textarabic{يبدو أن هناك بطة في الفناء الخلفي}
\end{flushright}

\begin{center}
        \hfill\textarabic{٧}\hfill\textarabic{٦}\hfill\textarabic{٥}\hfill\textarabic{٤}\hfill\textarabic{٣}\hfill\textarabic{٢}\hfill\textarabic{١}
        \end{center}


\vspace{0.5\baselineskip}\begin{flushright}
\textarabic{ماذا تشتكي إذا عزف الموسيقيّ في الحفل؟}
\end{flushright}

\begin{center}
        \hfill\textarabic{٧}\hfill\textarabic{٦}\hfill\textarabic{٥}\hfill\textarabic{٤}\hfill\textarabic{٣}\hfill\textarabic{٢}\hfill\textarabic{١}
        \end{center}
		
\vfill\clearpage

\vspace{0.5\baselineskip}\begin{flushright}
\textarabic{لا يمكن تصور سعاد لتصل على الوقت}
\end{flushright}

\begin{center}
        \hfill\textarabic{٧}\hfill\textarabic{٦}\hfill\textarabic{٥}\hfill\textarabic{٤}\hfill\textarabic{٣}\hfill\textarabic{٢}\hfill\textarabic{١}
        \end{center}


\vspace{0.5\baselineskip}\begin{flushright}
\textarabic{مريم موهوبة أكثر من كونها ذكية}
\end{flushright}

\begin{center}
        \hfill\textarabic{٧}\hfill\textarabic{٦}\hfill\textarabic{٥}\hfill\textarabic{٤}\hfill\textarabic{٣}\hfill\textarabic{٢}\hfill\textarabic{١}
        \end{center}


\vspace{0.5\baselineskip}\begin{flushright}
\textarabic{فاطمة تساءلت ماذا من طلب من المطعم}
\end{flushright}

\begin{center}
        \hfill\textarabic{٧}\hfill\textarabic{٦}\hfill\textarabic{٥}\hfill\textarabic{٤}\hfill\textarabic{٣}\hfill\textarabic{٢}\hfill\textarabic{١}
        \end{center}


\vspace{0.5\baselineskip}\begin{flushright}
\textarabic{أقنعت موزة علي كونها ذهبت ألى الحفلة}
\end{flushright}

\begin{center}
        \hfill\textarabic{٧}\hfill\textarabic{٦}\hfill\textarabic{٥}\hfill\textarabic{٤}\hfill\textarabic{٣}\hfill\textarabic{٢}\hfill\textarabic{١}
        \end{center}


\vspace{0.5\baselineskip}\begin{flushright}
\textarabic{قرأ اشياءً، ماجد قد فعل بسرعة}
\end{flushright}

\begin{center}
        \hfill\textarabic{٧}\hfill\textarabic{٦}\hfill\textarabic{٥}\hfill\textarabic{٤}\hfill\textarabic{٣}\hfill\textarabic{٢}\hfill\textarabic{١}
        \end{center}


\vspace{0.5\baselineskip}\begin{flushright}
\textarabic{بعضهم البعض احتضنت خديجة و عائشة}
\end{flushright}

\begin{center}
        \hfill\textarabic{٧}\hfill\textarabic{٦}\hfill\textarabic{٥}\hfill\textarabic{٤}\hfill\textarabic{٣}\hfill\textarabic{٢}\hfill\textarabic{١}
        \end{center}


\vspace{0.5\baselineskip}\begin{flushright}
\textarabic{من يتساءل ما إذا كانت منى باعت التلفاز؟}
\end{flushright}

\begin{center}
        \hfill\textarabic{٧}\hfill\textarabic{٦}\hfill\textarabic{٥}\hfill\textarabic{٤}\hfill\textarabic{٣}\hfill\textarabic{٢}\hfill\textarabic{١}
        \end{center}


\vspace{0.5\baselineskip}\begin{flushright}
\textarabic{التقرير حول الكتاب أخاب ظن سامي في التلاميذ}
\end{flushright}

\begin{center}
        \hfill\textarabic{٧}\hfill\textarabic{٦}\hfill\textarabic{٥}\hfill\textarabic{٤}\hfill\textarabic{٣}\hfill\textarabic{٢}\hfill\textarabic{١}
        \end{center}


\vspace{0.5\baselineskip}\begin{flushright}
\textarabic{من يخجل إذا رسم الفنان لوحة اليوم؟}
\end{flushright}

\begin{center}
        \hfill\textarabic{٧}\hfill\textarabic{٦}\hfill\textarabic{٥}\hfill\textarabic{٤}\hfill\textarabic{٣}\hfill\textarabic{٢}\hfill\textarabic{١}
        \end{center}


\vspace{0.5\baselineskip}\begin{flushright}
\textarabic{قال احمد لفاطمة أن تعتني بنفسها}
\end{flushright}

\begin{center}
        \hfill\textarabic{٧}\hfill\textarabic{٦}\hfill\textarabic{٥}\hfill\textarabic{٤}\hfill\textarabic{٣}\hfill\textarabic{٢}\hfill\textarabic{١}
        \end{center}

\vfill\clearpage

\vspace{0.5\baselineskip}\begin{flushright}
\textarabic{ماذا تعتقد أن المحامي نسي في المكتب؟}
\end{flushright}

\begin{center}
        \hfill\textarabic{٧}\hfill\textarabic{٦}\hfill\textarabic{٥}\hfill\textarabic{٤}\hfill\textarabic{٣}\hfill\textarabic{٢}\hfill\textarabic{١}
        \end{center}


\vspace{0.5\baselineskip}\begin{flushright}
\textarabic{تفحصت عائشة الأشياء و فعلت بدقة ذلك}
\end{flushright}

\begin{center}
        \hfill\textarabic{٧}\hfill\textarabic{٦}\hfill\textarabic{٥}\hfill\textarabic{٤}\hfill\textarabic{٣}\hfill\textarabic{٢}\hfill\textarabic{١}
        \end{center}


\vspace{0.5\baselineskip}\begin{flushright}
\textarabic{تحدث مازن لرجل طويل القامة مثل والده}
\end{flushright}

\begin{center}
        \hfill\textarabic{٧}\hfill\textarabic{٦}\hfill\textarabic{٥}\hfill\textarabic{٤}\hfill\textarabic{٣}\hfill\textarabic{٢}\hfill\textarabic{١}
        \end{center}


\vspace{0.5\baselineskip}\begin{flushright}
\textarabic{تجادل سيف مع رجل عنيد مثل أخيه}
\end{flushright}

\begin{center}
        \hfill\textarabic{٧}\hfill\textarabic{٦}\hfill\textarabic{٥}\hfill\textarabic{٤}\hfill\textarabic{٣}\hfill\textarabic{٢}\hfill\textarabic{١}
        \end{center}


\vspace{0.5\baselineskip}\begin{flushright}
\textarabic{كم عدد الكتب هناك كانت على الطاولة؟}
\end{flushright}

\begin{center}
        \hfill\textarabic{٧}\hfill\textarabic{٦}\hfill\textarabic{٥}\hfill\textarabic{٤}\hfill\textarabic{٣}\hfill\textarabic{٢}\hfill\textarabic{١}
        \end{center}


\vspace{0.5\baselineskip}\begin{flushright}
\textarabic{يبدو امراة تكون في الفناء}
\end{flushright}

\begin{center}
        \hfill\textarabic{٧}\hfill\textarabic{٦}\hfill\textarabic{٥}\hfill\textarabic{٤}\hfill\textarabic{٣}\hfill\textarabic{٢}\hfill\textarabic{١}
        \end{center}


\vspace{0.5\baselineskip}\begin{flushright}
\textarabic{لُعبت اللعبة بدون أحذية}
\end{flushright}

\begin{center}
        \hfill\textarabic{٧}\hfill\textarabic{٦}\hfill\textarabic{٥}\hfill\textarabic{٤}\hfill\textarabic{٣}\hfill\textarabic{٢}\hfill\textarabic{١}
        \end{center}


\vspace{0.5\baselineskip}\begin{flushright}
\textarabic{سألت فدوى ماذا من اشترى}
\end{flushright}

\begin{center}
        \hfill\textarabic{٧}\hfill\textarabic{٦}\hfill\textarabic{٥}\hfill\textarabic{٤}\hfill\textarabic{٣}\hfill\textarabic{٢}\hfill\textarabic{١}
        \end{center}


\vspace{0.5\baselineskip}\begin{flushright}
\textarabic{عصام أكثر طولا من لائق}
\end{flushright}

\begin{center}
        \hfill\textarabic{٧}\hfill\textarabic{٦}\hfill\textarabic{٥}\hfill\textarabic{٤}\hfill\textarabic{٣}\hfill\textarabic{٢}\hfill\textarabic{١}
        \end{center}


\vspace{0.5\baselineskip}\begin{flushright}
\textarabic{سمر أرادت أن تعطي التلاميذ شيأً جيداً الكتابة عنه، و تعطي التلاميذ مجموعة عظيمة فعلتها من المواضيع}
\end{flushright}

\begin{center}
        \hfill\textarabic{٧}\hfill\textarabic{٦}\hfill\textarabic{٥}\hfill\textarabic{٤}\hfill\textarabic{٣}\hfill\textarabic{٢}\hfill\textarabic{١}
        \end{center}

\vfill\clearpage

\vspace{0.5\baselineskip}\begin{flushright}
\textarabic{حاول جمال أن يفوز}
\end{flushright}

\begin{center}
        \hfill\textarabic{٧}\hfill\textarabic{٦}\hfill\textarabic{٥}\hfill\textarabic{٤}\hfill\textarabic{٣}\hfill\textarabic{٢}\hfill\textarabic{١}
        \end{center}


\vspace{0.5\baselineskip}\begin{flushright}
\textarabic{جمال يريد لكل واحد تريده ان يستمتع}
\end{flushright}

\begin{center}
        \hfill\textarabic{٧}\hfill\textarabic{٦}\hfill\textarabic{٥}\hfill\textarabic{٤}\hfill\textarabic{٣}\hfill\textarabic{٢}\hfill\textarabic{١}
        \end{center}


\vspace{0.5\baselineskip}\begin{flushright}
\textarabic{حفصة أرادت أن تعطي الخيرية شيأً دافئاً تلبسه، و تعطي الخيرية مجموعة من السترات هي ما قد فعلت}
\end{flushright}

\begin{center}
        \hfill\textarabic{٧}\hfill\textarabic{٦}\hfill\textarabic{٥}\hfill\textarabic{٤}\hfill\textarabic{٣}\hfill\textarabic{٢}\hfill\textarabic{١}
        \end{center}


\vspace{0.5\baselineskip}\begin{flushright}
\textarabic{أدرك محمد أن زوجته تكره أن تُخاطب بعنف أمام الأطفال}
\end{flushright}

\begin{center}
        \hfill\textarabic{٧}\hfill\textarabic{٦}\hfill\textarabic{٥}\hfill\textarabic{٤}\hfill\textarabic{٣}\hfill\textarabic{٢}\hfill\textarabic{١}
        \end{center}


\vspace{0.5\baselineskip}\begin{flushright}
\textarabic{مشينا بضع دقائق، ثم جلسنا على المقعد}
\end{flushright}

\begin{center}
        \hfill\textarabic{٧}\hfill\textarabic{٦}\hfill\textarabic{٥}\hfill\textarabic{٤}\hfill\textarabic{٣}\hfill\textarabic{٢}\hfill\textarabic{١}
        \end{center}


\vspace{0.5\baselineskip}\begin{flushright}
\textarabic{الكلب رايت طوقه كان من جلد}
\end{flushright}

\begin{center}
        \hfill\textarabic{٧}\hfill\textarabic{٦}\hfill\textarabic{٥}\hfill\textarabic{٤}\hfill\textarabic{٣}\hfill\textarabic{٢}\hfill\textarabic{١}
        \end{center}


\vspace{0.5\baselineskip}\begin{flushright}
\textarabic{كم من المال هناك كانوا في حسابك؟}
\end{flushright}

\begin{center}
        \hfill\textarabic{٧}\hfill\textarabic{٦}\hfill\textarabic{٥}\hfill\textarabic{٤}\hfill\textarabic{٣}\hfill\textarabic{٢}\hfill\textarabic{١}
        \end{center}


\vspace{0.5\baselineskip}\begin{flushright}
\textarabic{كل الرجال يبدو أنهم تناولوا كلهم العشاء}
\end{flushright}

\begin{center}
        \hfill\textarabic{٧}\hfill\textarabic{٦}\hfill\textarabic{٥}\hfill\textarabic{٤}\hfill\textarabic{٣}\hfill\textarabic{٢}\hfill\textarabic{١}
        \end{center}


\vspace{0.5\baselineskip}\begin{flushright}
\textarabic{ماذا يعتقد الشرطي أن محمود سرق؟}
\end{flushright}

\begin{center}
        \hfill\textarabic{٧}\hfill\textarabic{٦}\hfill\textarabic{٥}\hfill\textarabic{٤}\hfill\textarabic{٣}\hfill\textarabic{٢}\hfill\textarabic{١}
        \end{center}


\vspace{0.5\baselineskip}\begin{flushright}
\textarabic{هذه طاولةً}
\end{flushright}

\begin{center}
        \hfill\textarabic{٧}\hfill\textarabic{٦}\hfill\textarabic{٥}\hfill\textarabic{٤}\hfill\textarabic{٣}\hfill\textarabic{٢}\hfill\textarabic{١}
        \end{center}

\vfill\clearpage

\vspace{0.5\baselineskip}\begin{flushright}
\textarabic{انكسر الكأسُ الولدَ}
\end{flushright}

\begin{center}
        \hfill\textarabic{٧}\hfill\textarabic{٦}\hfill\textarabic{٥}\hfill\textarabic{٤}\hfill\textarabic{٣}\hfill\textarabic{٢}\hfill\textarabic{١}
        \end{center}


\vspace{0.5\baselineskip}\begin{flushright}
\textarabic{أخبرت السيد زياد أنني أتسائل حين نطلي السياج معا}
\end{flushright}

\begin{center}
        \hfill\textarabic{٧}\hfill\textarabic{٦}\hfill\textarabic{٥}\hfill\textarabic{٤}\hfill\textarabic{٣}\hfill\textarabic{٢}\hfill\textarabic{١}
        \end{center}


\vspace{0.5\baselineskip}\begin{flushright}
\textarabic{أخ و أخت الذين كانوا يلعبون طوال الوقت ارسالهم وُجِب الى السرير}
\end{flushright}

\begin{center}
        \hfill\textarabic{٧}\hfill\textarabic{٦}\hfill\textarabic{٥}\hfill\textarabic{٤}\hfill\textarabic{٣}\hfill\textarabic{٢}\hfill\textarabic{١}
        \end{center}


\vspace{0.5\baselineskip}\begin{flushright}
\textarabic{ماذا تظنّ المحققة أن مروة لاحقته؟}
\end{flushright}

\begin{center}
        \hfill\textarabic{٧}\hfill\textarabic{٦}\hfill\textarabic{٥}\hfill\textarabic{٤}\hfill\textarabic{٣}\hfill\textarabic{٢}\hfill\textarabic{١}
        \end{center}


\vspace{0.5\baselineskip}\begin{flushright}
\textarabic{هذه مروحة}
\end{flushright}

\begin{center}
        \hfill\textarabic{٧}\hfill\textarabic{٦}\hfill\textarabic{٥}\hfill\textarabic{٤}\hfill\textarabic{٣}\hfill\textarabic{٢}\hfill\textarabic{١}
        \end{center}


\vspace{0.5\baselineskip}\begin{flushright}
\textarabic{ماذا نشرتَ الادعاء أن علي سرقه؟}
\end{flushright}

\begin{center}
        \hfill\textarabic{٧}\hfill\textarabic{٦}\hfill\textarabic{٥}\hfill\textarabic{٤}\hfill\textarabic{٣}\hfill\textarabic{٢}\hfill\textarabic{١}
        \end{center}


\vspace{0.5\baselineskip}\begin{flushright}
\textarabic{حامد يأمل للجميع كما تفعل النجاح}
\end{flushright}

\begin{center}
        \hfill\textarabic{٧}\hfill\textarabic{٦}\hfill\textarabic{٥}\hfill\textarabic{٤}\hfill\textarabic{٣}\hfill\textarabic{٢}\hfill\textarabic{١}
        \end{center}


\vspace{0.5\baselineskip}\begin{flushright}
\textarabic{هناك كان مشتبه به أُعتُبِر مذنب}
\end{flushright}

\begin{center}
        \hfill\textarabic{٧}\hfill\textarabic{٦}\hfill\textarabic{٥}\hfill\textarabic{٤}\hfill\textarabic{٣}\hfill\textarabic{٢}\hfill\textarabic{١}
        \end{center}


\vspace{0.5\baselineskip}\begin{flushright}
\textarabic{هي تتساءل حول من القصة التي في الصفحة الأمامية}
\end{flushright}

\begin{center}
        \hfill\textarabic{٧}\hfill\textarabic{٦}\hfill\textarabic{٥}\hfill\textarabic{٤}\hfill\textarabic{٣}\hfill\textarabic{٢}\hfill\textarabic{١}
        \end{center}


\vspace{0.5\baselineskip}\begin{flushright}
\textarabic{ماذا سمعتَ أن ميرا حضّرت؟}
\end{flushright}

\begin{center}
        \hfill\textarabic{٧}\hfill\textarabic{٦}\hfill\textarabic{٥}\hfill\textarabic{٤}\hfill\textarabic{٣}\hfill\textarabic{٢}\hfill\textarabic{١}
        \end{center}

\vfill\clearpage

\vspace{0.5\baselineskip}\begin{flushright}
\textarabic{حاولت مريم أن تركض في الماراثون}
\end{flushright}

\begin{center}
        \hfill\textarabic{٧}\hfill\textarabic{٦}\hfill\textarabic{٥}\hfill\textarabic{٤}\hfill\textarabic{٣}\hfill\textarabic{٢}\hfill\textarabic{١}
        \end{center}


\vspace{0.5\baselineskip}\begin{flushright}
\textarabic{يبدو له أن كريماً حل المشكلة}
\end{flushright}

\begin{center}
        \hfill\textarabic{٧}\hfill\textarabic{٦}\hfill\textarabic{٥}\hfill\textarabic{٤}\hfill\textarabic{٣}\hfill\textarabic{٢}\hfill\textarabic{١}
        \end{center}


\vspace{0.5\baselineskip}\begin{flushright}
\textarabic{يبدو لهم أن سليماً هو الشخص المناسب للعمل}
\end{flushright}

\begin{center}
        \hfill\textarabic{٧}\hfill\textarabic{٦}\hfill\textarabic{٥}\hfill\textarabic{٤}\hfill\textarabic{٣}\hfill\textarabic{٢}\hfill\textarabic{١}
        \end{center}


\vspace{0.5\baselineskip}\begin{flushright}
\textarabic{من الذي الخادمة تعتبر جار مشبوه}
\end{flushright}

\begin{center}
        \hfill\textarabic{٧}\hfill\textarabic{٦}\hfill\textarabic{٥}\hfill\textarabic{٤}\hfill\textarabic{٣}\hfill\textarabic{٢}\hfill\textarabic{١}
        \end{center}


\vspace{0.5\baselineskip}\begin{flushright}
\textarabic{ذهبت ليلى إلى القاهرة، وظن سالم أن ميرا ذهبت أيضاً}
\end{flushright}

\begin{center}
        \hfill\textarabic{٧}\hfill\textarabic{٦}\hfill\textarabic{٥}\hfill\textarabic{٤}\hfill\textarabic{٣}\hfill\textarabic{٢}\hfill\textarabic{١}
        \end{center}


\vspace{0.5\baselineskip}\begin{flushright}
\textarabic{سيف أراد ان يعطي الوالدين شيئا خاصا لتذكر احتفال التخرج، و يعطي الوالدين لوحة كبيرة من صنعه}
\end{flushright}

\begin{center}
        \hfill\textarabic{٧}\hfill\textarabic{٦}\hfill\textarabic{٥}\hfill\textarabic{٤}\hfill\textarabic{٣}\hfill\textarabic{٢}\hfill\textarabic{١}
        \end{center}


\vspace{0.5\baselineskip}\begin{flushright}
\textarabic{الشوكة ذات المقبض الفضي أطراف حادة}
\end{flushright}

\begin{center}
        \hfill\textarabic{٧}\hfill\textarabic{٦}\hfill\textarabic{٥}\hfill\textarabic{٤}\hfill\textarabic{٣}\hfill\textarabic{٢}\hfill\textarabic{١}
        \end{center}


\vspace{0.5\baselineskip}\begin{flushright}
\textarabic{ماذا تضحك إذا كتبه الملك بالأمس؟}
\end{flushright}

\begin{center}
        \hfill\textarabic{٧}\hfill\textarabic{٦}\hfill\textarabic{٥}\hfill\textarabic{٤}\hfill\textarabic{٣}\hfill\textarabic{٢}\hfill\textarabic{١}
        \end{center}


\vspace{0.5\baselineskip}\begin{flushright}
\textarabic{يبدو أن رجل يكون في الغرفة}
\end{flushright}

\begin{center}
        \hfill\textarabic{٧}\hfill\textarabic{٦}\hfill\textarabic{٥}\hfill\textarabic{٤}\hfill\textarabic{٣}\hfill\textarabic{٢}\hfill\textarabic{١}
        \end{center}


\vspace{0.5\baselineskip}\begin{flushright}
\textarabic{جمال قصد ان يعطي الاطفال شيأً جميلاً لياْكل، و يعطى الاطفال حفنة سخية من الحلوى قد فعل}
\end{flushright}

\begin{center}
        \hfill\textarabic{٧}\hfill\textarabic{٦}\hfill\textarabic{٥}\hfill\textarabic{٤}\hfill\textarabic{٣}\hfill\textarabic{٢}\hfill\textarabic{١}
        \end{center}

\vfill\clearpage

\vspace{0.5\baselineskip}\begin{flushright}
\textarabic{هذا هو كتاب}
\end{flushright}

\begin{center}
        \hfill\textarabic{٧}\hfill\textarabic{٦}\hfill\textarabic{٥}\hfill\textarabic{٤}\hfill\textarabic{٣}\hfill\textarabic{٢}\hfill\textarabic{١}
        \end{center}


\vspace{0.5\baselineskip}\begin{flushright}
\textarabic{ماذا يتساءل المصرفي ما إذا كان ناصر اشتراه؟}
\end{flushright}

\begin{center}
        \hfill\textarabic{٧}\hfill\textarabic{٦}\hfill\textarabic{٥}\hfill\textarabic{٤}\hfill\textarabic{٣}\hfill\textarabic{٢}\hfill\textarabic{١}
        \end{center}


\vspace{0.5\baselineskip}\begin{flushright}
\textarabic{من يظنّ أن نور قرأت الكتاب؟}
\end{flushright}

\begin{center}
        \hfill\textarabic{٧}\hfill\textarabic{٦}\hfill\textarabic{٥}\hfill\textarabic{٤}\hfill\textarabic{٣}\hfill\textarabic{٢}\hfill\textarabic{١}
        \end{center}


\vspace{0.5\baselineskip}\begin{flushright}
\textarabic{ماذا أنكرتَ أن احمد أكله؟}
\end{flushright}

\begin{center}
        \hfill\textarabic{٧}\hfill\textarabic{٦}\hfill\textarabic{٥}\hfill\textarabic{٤}\hfill\textarabic{٣}\hfill\textarabic{٢}\hfill\textarabic{١}
        \end{center}


\vspace{0.5\baselineskip}\begin{flushright}
\textarabic{قط و كلب كانوا يتعاركون طوال لوقت يجب فصلهم}
\end{flushright}

\begin{center}
        \hfill\textarabic{٧}\hfill\textarabic{٦}\hfill\textarabic{٥}\hfill\textarabic{٤}\hfill\textarabic{٣}\hfill\textarabic{٢}\hfill\textarabic{١}
        \end{center}


\vspace{0.5\baselineskip}\begin{flushright}
\textarabic{بعضهم البعض يحب علي و سلمان}
\end{flushright}

\begin{center}
        \hfill\textarabic{٧}\hfill\textarabic{٦}\hfill\textarabic{٥}\hfill\textarabic{٤}\hfill\textarabic{٣}\hfill\textarabic{٢}\hfill\textarabic{١}
        \end{center}


\vspace{0.5\baselineskip}\begin{flushright}
\textarabic{هذا قلم}
\end{flushright}

\begin{center}
        \hfill\textarabic{٧}\hfill\textarabic{٦}\hfill\textarabic{٥}\hfill\textarabic{٤}\hfill\textarabic{٣}\hfill\textarabic{٢}\hfill\textarabic{١}
        \end{center}


\vspace{0.5\baselineskip}\begin{flushright}
\textarabic{خبزت احمد فطيرة}
\end{flushright}

\begin{center}
        \hfill\textarabic{٧}\hfill\textarabic{٦}\hfill\textarabic{٥}\hfill\textarabic{٤}\hfill\textarabic{٣}\hfill\textarabic{٢}\hfill\textarabic{١}
        \end{center}


\vspace{0.5\baselineskip}\begin{flushright}
\textarabic{يبدو بأن وزيراً ما سوف يستقيل، لكن اي وزير لايزال سراً}
\end{flushright}

\begin{center}
        \hfill\textarabic{٧}\hfill\textarabic{٦}\hfill\textarabic{٥}\hfill\textarabic{٤}\hfill\textarabic{٣}\hfill\textarabic{٢}\hfill\textarabic{١}
        \end{center}


\vspace{0.5\baselineskip}\begin{flushright}
\textarabic{هناك يبدو أن رجل في الحديقة}
\end{flushright}

\begin{center}
        \hfill\textarabic{٧}\hfill\textarabic{٦}\hfill\textarabic{٥}\hfill\textarabic{٤}\hfill\textarabic{٣}\hfill\textarabic{٢}\hfill\textarabic{١}
        \end{center}

\vfill\clearpage

\vspace{0.5\baselineskip}\begin{flushright}
\textarabic{ماذا أعلنتَ الخبر أن فاطمة خسرت؟}
\end{flushright}

\begin{center}
        \hfill\textarabic{٧}\hfill\textarabic{٦}\hfill\textarabic{٥}\hfill\textarabic{٤}\hfill\textarabic{٣}\hfill\textarabic{٢}\hfill\textarabic{١}
        \end{center}


\vspace{0.5\baselineskip}\begin{flushright}
\textarabic{يبدو أن هناك موظفاً ما سوف يفشي المعلومات، و لكن لايزال ذلك الموظف مجهولاً}
\end{flushright}

\begin{center}
        \hfill\textarabic{٧}\hfill\textarabic{٦}\hfill\textarabic{٥}\hfill\textarabic{٤}\hfill\textarabic{٣}\hfill\textarabic{٢}\hfill\textarabic{١}
        \end{center}


\vspace{0.5\baselineskip}\begin{flushright}
\textarabic{سلمى صاحت على علي ان يهتم بنفسه}
\end{flushright}

\begin{center}
        \hfill\textarabic{٧}\hfill\textarabic{٦}\hfill\textarabic{٥}\hfill\textarabic{٤}\hfill\textarabic{٣}\hfill\textarabic{٢}\hfill\textarabic{١}
        \end{center}


\vspace{0.5\baselineskip}\begin{flushright}
\textarabic{الكتاب انكتب الكتاب}
\end{flushright}

\begin{center}
        \hfill\textarabic{٧}\hfill\textarabic{٦}\hfill\textarabic{٥}\hfill\textarabic{٤}\hfill\textarabic{٣}\hfill\textarabic{٢}\hfill\textarabic{١}
        \end{center}


\vspace{0.5\baselineskip}\begin{flushright}
\textarabic{ناموا لمدة ساعة ثم ذهبوا إلى العمل}
\end{flushright}

\begin{center}
        \hfill\textarabic{٧}\hfill\textarabic{٦}\hfill\textarabic{٥}\hfill\textarabic{٤}\hfill\textarabic{٣}\hfill\textarabic{٢}\hfill\textarabic{١}
        \end{center}


\vspace{0.5\baselineskip}\begin{flushright}
\textarabic{ماذا يعرف الجندي ما إذا كان مصطفى كتب؟}
\end{flushright}

\begin{center}
        \hfill\textarabic{٧}\hfill\textarabic{٦}\hfill\textarabic{٥}\hfill\textarabic{٤}\hfill\textarabic{٣}\hfill\textarabic{٢}\hfill\textarabic{١}
        \end{center}


\vspace{0.5\baselineskip}\begin{flushright}
\textarabic{ماذا تتمنى أن الامير اشتراه في السوق؟}
\end{flushright}

\begin{center}
        \hfill\textarabic{٧}\hfill\textarabic{٦}\hfill\textarabic{٥}\hfill\textarabic{٤}\hfill\textarabic{٣}\hfill\textarabic{٢}\hfill\textarabic{١}
        \end{center}


\vspace{0.5\baselineskip}\begin{flushright}
\textarabic{المقال أغضب مصطفى في الحكومة}
\end{flushright}

\begin{center}
        \hfill\textarabic{٧}\hfill\textarabic{٦}\hfill\textarabic{٥}\hfill\textarabic{٤}\hfill\textarabic{٣}\hfill\textarabic{٢}\hfill\textarabic{١}
        \end{center}


\vspace{0.5\baselineskip}\begin{flushright}
\textarabic{من اعتقدت صديقاً له راض؟}
\end{flushright}

\begin{center}
        \hfill\textarabic{٧}\hfill\textarabic{٦}\hfill\textarabic{٥}\hfill\textarabic{٤}\hfill\textarabic{٣}\hfill\textarabic{٢}\hfill\textarabic{١}
        \end{center}


\vspace{0.5\baselineskip}\begin{flushright}
\textarabic{من أعلن الخبر أن عصام ربح الجائزة؟}
\end{flushright}

\begin{center}
        \hfill\textarabic{٧}\hfill\textarabic{٦}\hfill\textarabic{٥}\hfill\textarabic{٤}\hfill\textarabic{٣}\hfill\textarabic{٢}\hfill\textarabic{١}
        \end{center}


\vspace{0.5\baselineskip}\begin{flushright}
\textarabic{من غير الوارد هزاع ليعمل بجد}
\end{flushright}

\begin{center}
        \hfill\textarabic{٧}\hfill\textarabic{٦}\hfill\textarabic{٥}\hfill\textarabic{٤}\hfill\textarabic{٣}\hfill\textarabic{٢}\hfill\textarabic{١}
        \end{center}


\vspace{0.5\baselineskip}}



\vfill
\end{document}